%%%%%%%%%%%%%%%%%%%%%%%%%%%%%%%%%%%%%%%%%%%%%%%%%%%%%%%%%%%%%%%%%%%%%%%%%%%%%%%
% A clean template for an academic CV
%
% Uses tabularx to create two column entries (date and job/edu/citation).
% Defines commands to make adding entries simpler.
%
%%%%%%%%%%%%%%%%%%%%%%%%%%%%%%%%%%%%%%%%%%%%%%%%%%%%%%%%%%%%%%%%%%%%%%%%%%%%%%%

\documentclass[10pt, a4paper]{article}

\usepackage{xspace}

% Useful aliases
\newcommand{\LUSCC}{School of Computer and Communications}
\newcommand{\LUFST}{Faculty of Science and Technology }
\newcommand{\LU}{Lancaster University}
\newcommand{\UOCCL}{Computer Laboratory}
\newcommand{\UOC}{University of Cambridge}

% Identifying information
\newcommand{\Title}{Curriculum Vit\ae}
\newcommand{\FirstName}{Charalampos}
\newcommand{\LastName}{Rotsos}
\newcommand{\Initials}{C}
\newcommand{\MyName}{\FirstName\ \LastName}
\newcommand{\Me}{\textbf{\Initials.~\xspace\LastName}}  % For citations
\newcommand{\Email}{c.rotsos@lancaster.ac.uk}
\newcommand{\Website}{crotsos.github.io}
\newcommand{\Lab}{lancaster.ac.uk/scc/research/networking/}
\newcommand{\ORCID}{0000-0003-0252-9373}
\newcommand{\Affiliation}{\LUSCC \\ \LUFST \\ \LU}
\newcommand{\Address}{ InfoLab21}

% Template configuration
%%%%%%%%%%%%%%%%%%%%%%%%%%%%%%%%%%%%%%%%%%%%%%%%%%%%%%%%%%%%%%%%%%%%%%%%%%%%%%%

% Disable hyphenation
\usepackage[none]{hyphenat}

% Control the font size
\usepackage{anyfontsize}

% Template variables for styling
\newcommand{\TablePad}{\vspace{-0.4cm}}
\newcommand{\SoftwareTitle}[1]{{\bfseries #1}}
\newcommand{\TableTitle}[1]{{\fontsize{12pt}{0}\selectfont \itshape #1}}
\newcommand{\Invited}{\textbf{[Invited]}}

% For fancy and multipage tables
\usepackage{tabularx}
\usepackage{ltablex}

% Define a new environment to place all CV entries in a 2-column table.
% Left column are the dates, right column the entries.
\usepackage{environ}
\NewEnviron{EntriesTable}{
\TablePad
\begin{tabularx}{\textwidth}{@{}p{0.1\textwidth}@{\hspace{0.02\textwidth}}p{0.88\textwidth}@{}}
  \BODY
\end{tabularx}
}

\usepackage{environ}
\NewEnviron{EntriesList}{
\TablePad
\begin{tabularx}{\textwidth}{@{}p{\textwidth}@{}}
  \BODY
\end{tabularx}
}

% Macros to add links
\newcommand{\DOI}[1]{doi:\href{https://doi.org/#1}{#1}}
\newcommand{\Preprint}[1]{preprint: \href{https://doi.org/#1}{doi.org/#1}}
\newcommand{\Youtube}[1]{recording: \href{https://youtu.be/#1}{youtu.be/#1}}

% Tags to mark publications
%\newcommand{\OA}{{\bfseries [open access]}}
\newcommand{\OA}{}

% Macros to set the year and duration on the left column
\newcommand{\Duration}[2]{\fontsize{10pt}{0}\selectfont #1--#2}
\newcommand{\Year}[1]{\fontsize{10pt}{0}\selectfont #1}
\newcommand{\Ongoing}{}
%\newcommand{\Ongoing}{$\ast$}
\newcommand{\Future}{future}
\newcommand{\Review}{in review}
\newcommand{\Accepted}{accepted}
\newcommand{\Appointment}[3]{\textbf{#1} \newline #2, #3 \newline}

% Define command to insert month name and year as date
\usepackage{datetime}
\newdateformat{monthyear}{\monthname[\THEMONTH], \THEYEAR}

% Set the page margins
\usepackage[left=1in,right=1in,top=1in,bottom=1in]{geometry}

% No indentation
\setlength\parindent{0cm}

% Increase the line spacing
\renewcommand{\baselinestretch}{1.1}
% and the spacing between rows in tables
\renewcommand{\arraystretch}{1.5}

% Remove space between items in itemize and enumerate
\usepackage{enumitem}
\setlist{nosep}

% Use custom colors
\usepackage[usenames,dvipsnames]{xcolor}

% Set fonts. Requires compilation with xelatex
\usepackage{fontspec}  % required to make older xelatex compile with UTF8

% Configure the font style for sections
\usepackage{sectsty}
\sectionfont{\vspace{0.5cm}\bfseries\fontsize{12pt}{0}\selectfont\uppercase}
\subsectionfont{\vspace{0.2cm}\mdseries\fontsize{12pt}{0}\selectfont\uppercase}

% Set the spacing for sections
%\usepackage{titlesec}
%\titlespacing{\section}{0pt}{0cm}{0.3cm}
%\titlespacing{\subsection}{0pt}{0.3cm}{0.3cm}

% Disable number of sections. Use this instead of "section*" so that the sections still
% appear as PDF bookmarks. Otherwise, would have to add the table of contents entries
% manually.
\makeatletter
\renewcommand{\@seccntformat}[1]{}
\makeatother

% Set fancy headers
\usepackage{fancyhdr}
\pagestyle{fancy}
\fancyhf{}
\chead{
  \fontsize{10pt}{12pt}\selectfont
  \MyName
  \hspace{0.2cm} -- \hspace{0.2cm}
  \Title
  \hspace{0.2cm} -- \hspace{0.2cm}
  \monthyear\today
}
\rhead{}
\cfoot{\fontsize{10pt}{0}\selectfont \thepage}
\renewcommand{\headrulewidth}{0pt}

% Metadata for the PDF output and control of hyperlinks
\usepackage[colorlinks=true]{hyperref}
\hypersetup{
  pdftitle={\MyName\ - \Title},
  pdfauthor={\MyName},
  linkcolor=blue,
  citecolor=blue,
  filecolor=black,
  urlcolor=MidnightBlue
}
%%%%%%%%%%%%%%%%%%%%%%%%%%%%%%%%%%%%%%%%%%%%%%%%%%%%%%%%%%%%%%%%%%%%%%%%%%%%%%%


\begin{document}

% No header for the first page
\thispagestyle{empty}

%%%%%%%%%%%%%%%%%%%%%%%%%%%%%%%%%%%%%%%%%%%%%%%%%%%%%%%%%%%%%%%%%%%%%%%%%%%%%%%
% HEADER
{\fontsize{24pt}{0}\selectfont\MyName}\\[-0.1cm]
\rule{\textwidth}{0.2pt}
\begin{minipage}[t]{0.595\textwidth}
  \Affiliation
  \\
  \Address
\end{minipage}
\begin{minipage}[t]{0.405\textwidth}
  \begin{flushright}
  Last updated: \monthyear\today
  \\
    ORCID: \href{https://orcid.org/\ORCID}{\ORCID}
    \\
    email: \href{mailto:\Email}{\Email}
    \\
    Website: \href{https://www.\Website}{\Website}
  \end{flushright}
\end{minipage}

%%%%%%%%%%%%%%%%%%%%%%%%%%%%%%%%%%%%%%%%%%%%%%%%%%%%%%%%%%%%%%%%%%%%%%%%%%%%%%%
\section{General Information}
\subsection{Professional Appointments}

\begin{EntriesTable}
  \Duration{2017}{}  &
  \Appointment{Lecturer}{\LUSCC}{\LU, UK}
{\it SCC.150,SCC.365 module convenor, EPSRC Initiate, Toucan, NG-CDI project Co-I, Part-I tutor.}
  \\
  \Duration{2015}{2017}  &
  \Appointment{Senior Research Associate -- EPSRC TOUCAN}{\LUSCC}{\LU, UK}
{\it Design and implement a technology-agnostic 5G network orchestration architecture.}
  \\
  \Duration{2013}{2015}  &
    \Appointment{Research Associate -- DARPA (MRC)2}{\UOCCL}{\UOC, UK}
{\it Port the Mirage OS framework to the CHERI CPU architecture.}
    \\
    \Year{2013} &
\Appointment{Research Intern -- DARPA (MRC)2}{Computer Science
Laboratory}{Stanford Research Institute, US}
{\it Develop a hybrid experimentation platform for cloud networks.}
    \\
    \Year{2011} &
    \Appointment{Research Intent -- EPRSRC Homework}{Horizon Institute}{Nottingham University, UK}
{\it Design a user-friendly home router architecture.}
    \\
    \Year{2010} &
\Appointment{Research Intern -- EU Ofelia}{Deutche Telekom}{Technical University
Berlin, Germany}
{\it Design a performance characterisation platform for OpenFlow devices.}
    \\
    \Duration{2007}{2009} &
    \Appointment{Research Assistant -- EU SKIER}{Department of Informatics and
    communications}{University of Athens}
    {\it Develop an IoT wildfire detection platform using image processing.}
\end{EntriesTable}


%%%%%%%%%%%%%%%%%%%%%%%%%%%%%%%%%%%%%%%%%%%%%%%%%%%%%%%%%%%%%%%%%%%%%%%%%%%%%%%
\subsection{Education}

\begin{EntriesTable}
\Duration{2018}{2019} & \textbf{PGCAP, Lancaster University, UK} \newline
Thesis: ``Improving gender equity in CS programs; towards an
inclusive SCC curriculum''
\\
  \Duration{2009}{2014}  &
  \textbf{PhD in Computer Science, University of Cambridge, UK} \newline
  Thesis: ``Scalable and Extensible Networks
with Software Defined Networking''
  \\
  \Duration{2006}{2007}  &
  \textbf{MSc in Data Communications, Networks and Distributed Systems, UCL, UK}
  \\
  \Duration{2001}{2006}  &
  \textbf{BSc in Computer Science, University of Piraeus, Greece}
\end{EntriesTable}

\section{Research}

% I am a member of the SCC networking group, a research group with more than 30 years of
% international reputation on programmable and resilient networked
% systems and multimedia delivery research. I joined the group in 2017 as a lecturer,   
% supporting Prof. D. Hutchison to manage major national research projects, while
% being a member of the 2021 REF panels and preparing for his recent retirement.
% My  research is predominantly experimental, involving testbed-based
% system design, implementation, evaluation and application, and targets
% real-world impact, through industry collaborations and standards
% specifications.

% My research  improve the efficiency and resilience of Future
% Internet infrastructures.  With my work I aim to combine the design of network
% algorithms
% models and architectures with the development of prototype systems and testbeds,
% and produce research outcomes that cover levels 1-3 in the TRL scale. In
% parallel, through collaborations with major international industrial partners and standardization
% bodies, I frequently downstream my research in real world system designs and
% network standards. My research efforts contribute to the ongoing international
% recognition of the Lancaster networking group as an expert on network
% management.

Communication networks play a vital role in our daily social lives and economic
activity, morphing into a national critical infrastructure. I develop
algorithms, prototype systems and architectures that improve the operational
efficiency and resilience of modern communication infrastructures and
enable the delivery of Future Internet services. My research expertise spans
across all network layers, from hardware design to
application, and allows me to develop systems that offer holistic optimization
mechanisms. In parallel, I research novel autonomic management mechanisms that
simplify network configuration, reduce operational cost and
improve overall energy efficiency in network infrastructures. 

I am a member of the SCC networking group, a research group with more than 30 years of
international reputation on programmable and resilient networked
systems and multimedia delivery research. My work
enriches the existing network management research agenda with new expertise on
network service orchestration for next generation mobile network
technologies (5G, 6G), a priority research area of national and international
funding bodies, like EPSRC, DCMS and H2020. I have an extensive research
leadership portfolio, leading research activities in three EPSRC research
projects, positions I acquired by
supporting Prof. D. Hutchison to manage his research portfolio, while stepping
down from his role and joining the 2021 REF panel. In
parallel, I collaborate with several national (BT, BBC) and international (NEC,
Ericsson) network vendors and operators, leading effective knowledge transfer
projects that downstream research into production and standards. 

\subsection{Research Projects}

\begin{EntriesTable}
    \Duration{2019}{2022} &
EPSRC ``Next Generation Converged Digital infrastructure (NG-CDI)'', (Amount:
£2.5M, LU: £950k), PI: N. Race, Co-I: {\bf C. Rotsos},
 {\it Manage: M. Bezahaf, W. Fantom, B. Simms, E. Davies}. \newline
This EPSRC prosperity partnership, in collaboration with BT, develops
technologies that enable 
automated and autonomic management of production network infrastructures. I lead the
research efforts of a team of researchers between Lancaster, Surrey and
Cambridge developing mechanisms for autonomous network management and testing.
The project develops prototype systems that
improve the operational resilience and efficiency (energy, cost) in BT infrastructures. 
 \\
    \Duration{2017}{2020} &
    EPSRC ``The UK Programmable Fixed and Mobile Internet Infrastructure
    (INITIATE)'',  (Amount: £1.6M, LU: £400k), PI: D. Hutchison, Co-I: \Me. Manage: S. Simpson, P. Mccherry. \newline
This EPRSC project developed the first open 5G experimentation testbed between
four UK universities (Lancaster, Bristol, KCL, Edinburgh). I coordinated teams
between Lancaster, Edinburgh, KCL, and Bristol in
developing a production network orchestration platform. The 
testbed was used by several UK-based SMEs and academic groups to evaluate 5G technologies and use-cases.\\
    
    \Duration{2015}{2021} &
EPSRC ``Towards Ultimate Convergence (TOUCAN)'', (Amount: £5.9M, LU: £800k), PI: D.
Hutchison, Co-I: \Me, {\it Manage: A. Magzub, L. Hill, A. Farshad}.\newline
This EPSRC project developed technology-agnostic 
convergence mechanisms for future network infrastructures (optical, wireless, visible
light, mobile, compute). I coordinated research teams between
Lancaster, Edinburgh and Bristol, in developing an intelligent service orchestration
platform. Our research efforts developed fundamental research on network modeling
and programmability, essential for the development for 5G technologies. 
\end{EntriesTable}


\subsection{Research Proposals}

\begin{EntriesTable}
    \Year{2022} &
EPSRC future communications systems early-stage federated hub call: ``Pantheon'' [Under Review].
    (Amount: £ 2m - LU: £50k) Co-I: N. Race, \textbf{C. Rotsos}. \newline
This EPSRC project proposal will create a multi-university research hub on the
topic of Network of Networks. I will lead a work package on mobile network management and
orchestration, in collaboration with UEA, QUB and UCL.
The research will organize several funding calls for small research projects and
contribute to shaping UK's roadmap towards 6G mobile systems.\\
~ & EPSRC future communications systems early-stage federated hub call: ``IncludeHUB'' [Under Review].
    (Amount: £ 2m - LU: £50k) Co-I: N. Race, \textbf{C. Rotsos}. \newline
This EPSRC project proposal will create a multi-university research hub on the
topic of cloud and distributed systems. I will lead a work package on 
cloud-edge continuum management, in collaboration with KCL, Loughborough and
Glasgow.
The research will organize several funding calls for small relevant projects and
develop technologies that enable recent AI advances in
network management mechanisms.\\
~ & DCMS Future Open Networks Research Challenge: ``Towards Ubiquitous 3D Open Resilient Network (TUDOR)'' [Under Review].
    (Amount: £ 12m - LU: £850k ) Co-I: N. Race, \textbf{C. Rotsos}. \newline
This DCMS project proposal will develop the fundamental research 
for the next generation of mobile networks (6G). The large project
consortium (13 partners) includes two major mobile network vendors (Ericsson,
Nokia), a cloud provider (Amazon) and all major UK mobile operators (BT, O2,
Vodafone). I will lead the design of the management layer of the project
and develop mechanisms that converge control across mobile and satellite infrastructures.
    \\
    ~ &
    DCMS Future Open Networks Research Challenge: ``Towards Holistic Openness in Future Mobile Networks'' [Under Review].
    (Amount: £4m - LU: £950k) Co-I: N. Race, \textbf{C. Rotsos}. \newline
This DCMS project proposal will develop the first national open experimentation
testbed for 6G mobile technologies. I will lead the design and
implementation of a management and monitoring services.
    \\
    \Year{2019} &
    H2020: ``Nestor'' [Not funded - Score 14/15].
    (Amount: £4m, LU: £650k)
    PI: D. Hutchison, Co-I: C. Rotsos, A. Marnerides.\newline
This project proposal would develop an ICT framework for the management
of distributed renewable energy infrastructures.
Lancaster university lea a work package on security and resilience. 
    \\
    \Year{2018} &
    H2020 Fed4Fire+ Open Call: ``LEMNOS'' [Not funded - Score 48/60].
    (LU: £40k) PI: \Me, Co-I: A. Farshad.
This small project proposal would develop mechanisms that integrate emerging
orchestration mechanisms with Cloud OS technologies.
    \\
    ~ &
    H2020 5GinFire Open Call: ``ARCADINS'' [Not funded - Score 21.5/25].
    (LU: £40k) PI: \Me, M. Broadbent.
This small project proposal would develop a monitoring framework using
lightweight virtualization technologies (Unikernels).
    \\
    \Year{2016} &
    H2020 MC-ETN: ``5G-Nordic'' [Not funded - Score 86\%].
    (Amount: £2m, LU: £200K) PI: Andreas Mauthe, Co-I: \Me.
This H2020 ETN project proposal aimed to develop an inter/multidisciplinary
training network on 5G technologies.
Lancaster university led a work package on 5G resilience. 
 \end{EntriesTable}

%%%%%%%%%%%%%%%%%%%%%%%%%%%%%%%%%%%%%%%%%%%%%%%%%%%%%%%%%%%%%%%%%%%%%%%%%%%%%%%
\subsection{Selected Papers}
\begin{EntriesList}

    R Oudin, G Antichi, C Rotsos, AW Moore, S Uhlig.
    OFLOPS-SUME and the art of switch characterization.
\emph{IEEE Journal on Selected Areas in Communications}, 2018, vol. 36, no. 12, pp.
2612-2620, Impact Factor: 17.01.  \DOI{10.1109/JSAC.2018.2871235} \newline
The paper summarizes the results of a long-running multi-dimensional study of
OpenFlow switch implementations. Early results of this work identified
the challenges to implement the OpenFlow protocol in hardware and influenced
research efforts on control plane consistency.
    \\
W. Fantom, P. Alcock, B. Simms, C. Rotsos and N. Race. A NEAT way to
test-driven network management, Acceptance Rate: 29.6\%. \emph{2022 IEEE/IFIP NOMS},
pp. 1-5, \DOI{10.1109/NOMS54207.2022.9789909}. \newline
NEAT is the first open-source automated testing platform for network DevOps.
Demos of the system were presented to various BT stakeholders,
as well as  the TMF AN group SDO. The resulting platform will be used to support
the 6G digital twin platform for the TUDOR project.
\\
    S. Simpson, A. Farshad, P. McCherry, A. Magzoub, W. Fantom, C. Rotsos,
    N. Race, D. Hutchison.
    DataPlane Broker: Open WAN control for multi-site service orchestration.
\emph{2019 IEEE NFV-SDN}, Acceptance rate: 32\%.
\DOI{10.1109/NFV-SDN47374.2019.9040084} \newline
The DataPlane Broker was the first open-source WAN platform to support
point-to-multipoint connectivity. A demo of the platform was accepted as a Proof
of Concept project by the ETSI NFV Special Interest Group, code of the platform
has been integrated with the ETSI OSM orchestrator source code and influenced
revisions in the NFV-MANO model to support multi-cloud service orchestration. 
    \\
     J.H. Han, P. Mundkur, C. Rotsos, G. Antichi, N. Dave, A.W. Moore, P.G.  Neumann.
     Blueswitch: Enabling provably consistent configuration of network switches.
\emph{ACM/IEEE ANCS}, pp. 17-27, \DOI{10.1109/ANCS.2015.7110117} (Acceptance
Rate: 28\%, 40 References) \newline
This paper presents a collaboration between Lancaster University, Stanford
Research Institute and Cambridge University in developing the first open source
OpenFlow switch architecture with built-in support for consistent updates with
strong correctness guarantees. The system influenced the field of network
programmability and open hardware. The work of this paper helped us to organize
two successful tutorial sessions in ACM Sigcomm 2015 and IEEE NetSoft 2015.
     \\
    C. Rotsos, D. King, A. Farshad, J. Bird, L. Fawcett, N. Georgalas, M. Gunkel, K. Shiomoto, A. Wang, A. Mauthe, N. Race, D. Hutchison.
    Network service orchestration standardization: A technology survey.
    \emph{Elsevier Computer Standards \& Interfaces}, vol. 54, no. 4,
pp. 203-215, 2018, \DOI{10.1016/j.csi.2016.12.006} (Impact Factor: 3.721, 88
references) \newline
This is the first survey paper to cover the topic of network service
orchestration in the field of network management. I collaborated with researchers
from several major international network operators, including BT,
Deutche Telekoms, China Telecoms and NTT. The survey strengthened the
international recognition of the SCC networking group expertise on network
orchestration and assisted us to join a number of standardization bodies. 
\end{EntriesList}

\subsection{Conferences \& Journals}
\begin{EntriesTable}
\Year{2022} &
A. Jung, H. Lefeuvre, C. Rotsos, P. Olivier, D.
Oñoro-Rubio, F. Huici, and M. Niepert. Zero-knowledge OS optimization for cloud
appliances. \emph{2023 USENIX OSDI.} (under development).
\\
~ &
W. Fantom, E. Davies, C. Rotsos, P. Veitch, S. Cassidy, and N. Race. NES: Towards lifecycle automation for
emulation-based experimentation. \emph{IEEE NOMS} (under review).
\\
~ &
B. Lewis, M. Broadbent, C. Rotsos and N. Race. 4MIDable: Flexible Network
Offloading For Security VNFs. \emph{Journal of Network and Systems Management.} (under review).
\\
~ &
P. Alcock, B. Simms, W. Fantom, C. Rotsos and N. Race. Improving Intent
Correctness with Automated Testing. \emph{2022 IEEE NetSoft}, pp. 61-66, 
\DOI{10.1109/NetSoft54395.2022.9844054}.
\\
~ &
L. Hill, C. Rotsos, W. Fantom, C. Edwards and D. Hutchison. Improving network resilience with Middlebox Minions. \emph{IEEE/IFIP NOMS}, pp. 1-5, \DOI{10.1109/NOMS54207.2022.9789819}. 
\OA
\\
~ &
N. Race, I. Eckley, A. Parlikad, C. Rotsos, N. Wang,
R. Piechocki, P. Stiles, A. Parekh, T. Burbridge, 
P. Willis,  and S. Cassidy. Industry-Academia Research toward Future
Network Intelligence: The NG-CDI Prosperity Partnership. \emph{in IEEE
Network}, vol. 36, no. 1, pp. 18-24, 
\DOI{10.1109/MNET.001.2100405}.
\\
\Year{2021} & 
A. Jung, H. Lefeuvre, C. Rotsos, P. Olivier, D.
Oñoro-Rubio, F. Huici, and M. Niepert. Wayfinder: towards
automatically deriving optimal OS configurations. \emph{ACM
SIGOPS Asia-Pacific Workshop on Systems.} pp. 115–122. \DOI{10.1145/3476886.3477506}.
\\
~ &
M. Bezahaf, S. Cassidy, D. Hutchison, D. King, N. Race, and C.  Rotsos. A Model-Driven and Business Approach to Autonomic Network Management.
\emph{Journal of ICT Standardization}, vol. 9, no. 2, 229-256, \DOI{10.13052/jicts2245-800X.928}.
\OA
\\
~ &
M. Bezahaf, E. Davies, C. Rotsos and N. Race, To All Intents and Purposes:
Towards Flexible Intent Expression, \emph{IEEE NetSoft}, pp. 31-37, \DOI{10.1109/NetSoft51509.2021.9492554}.
\OA
\\
    \Year{2020}  &
C. Rotsos, A. Marnerides, A. Magzoub, A. Jindal P. McCherry, M. Bor, J. Vidler,
D. Hutchison, Ukko: Resilient DRES management for Ancillary Services
using
5G service orchestration, \emph{IEEE SmartGridComm}, pp. 1-6, \DOI{10.1109/SmartGridComm47815.2020.9302980}.
\OA
    \\
    ~ & 
    H. Alshaer, N. Uniyal, K. Katsaros, K. Antonakoglou, S. Simpson, H. Abumarshoud,
    H. Falaki, P. McCherry, C. Rotsos, T. Mahmoodi, R. Nejabati, D.
    Kaleshi, D. Hutchison, H. Haas, D. Simeonidou.
    The UK Programmable Fixed and Mobile Internet Infrastructure: Overview,
    capabilities and use cases deployment.
    \emph{IEEE Access}, vol. 8, pp. 175398-175411,
    \DOI{10.1109/ACCESS.2020.3020894}
    \OA
    \\
    ~ &
    N Hart, C Rotsos, V Giotsas, N Race, D Hutchison.
    $\lambda$BGP: Rethinking BGP programmability.
    \emph{IEEE NOMS}, pp. 1-9,  \DOI{10.1109/NOMS47738.2020.9110331}
    \DOI{}
    \OA
    \\
    \Year{2018} &
    C. Rotsos, A. Farshad, D. King, D. Hutchison, Q. Zhou, A.J.G. Gray, C.X.
    Wang, S. McLaughlin.
    ReasoNet: Inferring network policies using ontologies.
    \emph{IEEE NetSoft},pp. 159-167, \DOI{10.1109/NETSOFT.2018.8460050} 
    \OA
    \\
    \Year{2017} &
    C. Rotsos, D. King, A. Farshad, J. Bird, L. Fawcett, N. Georgalas, M. Gunkel, K. Shiomoto, A. Wang, A. Mauthe, N. Race, D. Hutchison.
    Network service orchestration standardization: A technology survey.
    \emph{Elsevier Computer Standards \& Interfaces}, vol. 54, no. 4,
pp. 203-215, \DOI{10.1016/j.csi.2016.12.006}
    \OA
    \\
    ~ &
    D King, C Rotsos, I Busi, F Zhang, N Georgalas.
    Transport Northbound Interface: The Need for Specification and Standards
    Coordination.
    \emph{International Conference on Optical Network Design and Modeling}, \DOI{10.23919/ONDM.2017.7958527}
    \OA
     \\
     \Year{2016} &
     A. Chatzipapas, D. Pediaditakis, C. Rotsos, V.
     Mancuso, J. Crowcroft, A.W. Moore.
     Resolving data center power bill disputes: The energy-performance
     trade-offs of consolidation.
     \emph{ACM e-Energy}, pp. 89-94, \DOI{10.1145/2768510.2770933}
     \OA
     \\
     \Year{2015} &
     N. Zilberman, P.M. Watts, C. Rotsos, A.W. Moore.
     Reconfigurable network systems and software-defined networking.
     \emph{Proceesings of the IEEE}, vol. 103, no. 7, pp. 1102-1124, \DOI{10.1109/JPROC.2015.2435732}
     \OA
     \\
     ~ &
     C. Rotsos, G. Antichi, M. Bruyere, P. Owezarski, A.W. Moore.
     OLOPS-Turbo: Testing the next-generation OpenFlow switch.
     \emph{IEEE ICC}, pp. 5571-5576, \DOI{10.1109/ICC.2015.7249210}
     \OA
     \\
     \Year{2014} &
     D. Pediaditakis, C. Rotsos, A.W. Moore.
     Faithful reproduction of network experiments.
     \emph{ACM/IEEE ANCS}, pp. 41-52, \DOI{10.1145/2658260.2658274}
     \\
     \Year{2013} &
    A. Sathiaseelan, C. Rotsos, C.S. Sriram, D. Trossen,
    P. Papadimitriou, J. Crowcroft.
    Virtual public networks.
    \emph{EWSDN}, pp. 1-6, \DOI{10.1109/EWSDN.2013.7}
    \\
    ~ &
    C. Rotsos, H. Howard, D. Sheets, R. Mortier, A.
    Madhavapeddy, A. Chaudhry, J. Crowcroft.
    Lost in the edge: Finding your way with DNSSEC Signposts.
    \emph{USENIX FOCI}.
    \\
    ~ &
    A. Madhavapeddy, R. Mortier, C. Rotsos, D. Scott,
    B. Singh, T. Gazagnaire, S. Smith, S. Hand, J. Crowcroft.
    Unikernels: Library operating systems for the cloud.
    \emph{ACM ASPLOS}, pp. 461–472, 
    \DOI{10.1145/2490301.2451167}
    \\
    \Year{2012} &
    C. Rotsos, R. Mortier, A. Madhavapeddy, B. Singh, A.W. Moore.
    Cost, performance \& flexibility in openflow: Pick three.
    \emph{IEEE ICC}, pp. 6601-6605, \DOI{10.1109/ICC.2012.6364690}
    \\
    ~ &
    C. Rotsos, N. Sarrar, S. Uhlig, R. Sherwood, A.W. Moore.
    OFLOPS: An open framework for OpenFlow switch evaluation.
    \emph{International Conference on Passive and Active Measurement}. \DOI{10.1007/978-3-642-28537-0\_9}
    \\
    ~ &
    R. Mortier, T. Rodden, T. Lodge, D. McAuley, C. Rotsos,
    A. W. Moore, A. Koliousis, J. Sventek.
    Control and understanding: Owning your home network.
    \emph{COMSNETS}, pp. 1-10, \DOI{10.1109/COMSNETS.2012.6151322}.
    \\
    \Year{2011} &
    C. Rotsos, J. Van Gael, A.W. Moore, Z. Ghahramani.
    Probabilistic graphical models for semi-supervised traffic classification.
    \emph{Workshop on Traffic Analysis and Characterization}, 
    \DOI{10.1145/1815396.1815569}
    \\
\end{EntriesTable}

\subsection{Demos \& Posters}

\begin{EntriesTable}
    \Year{2015} &
    C. Rotsos, G. Antichi, A. W. Moore.
    Enabling Performance Evaluation Beyond 10 Gbps (DEMO).
    \emph{ACM SIGCOMM}, \DOI{10.1145/2785956.2790036}. 
    \\
    ~ &
    J. H. Han, G. Antichi, N. Zilberman, C. Rotsos, A. W. Moore.
    An integrated environment for open-source network softwarization.
    \emph{IEEE NetSoft}, \DOI{10.1109/NETSOFT.2015.7116167}.
    \\
    \Year{2014} &
    C. Rotsos, G. Antichi, M. Bruyere, P. Owezarski, A. W. Moore.
    An open testing framework for next-generation OpenFlow switches (Poster).
    \emph{EWSDN}, \DOI{10.1109/EWSDN.2014.12}.
    \\
    \Year{2012} &
    A. Chaudhry, A. Madhavapeddy, C. Rotsos, R. Mortier, A. Aucinas, J. Crowcroft, S. Probst Eide, S. Hand, A. W. Moore, N, Vallina-Rodriguez.
    Signposts: end-­to-­end networking in a world of middleboxes (DEMO).
    \emph{ACM SIGCOMM}, \DOI{10.1145/2377677.2377692}.
    \\
    \Year{2011} &
    R. Mortier, B. Bedwell, K. Glover, T. Lodge, T. Rodden, C. Rotsos, A. W. Moore, A. Koliousis, J. Sventek.
    Supporting novel home network management interfaces with OpenFlow and NOX.
    \emph{ACM SIGCOMM }, \DOI{10.1145/2018436.2018523}.
\end{EntriesTable}

\subsection{Awards \& Honors}

\begin{EntriesTable}
   \Year{2013}  &
    Best-paper award, ``Unikernels: library operating systems for the cloud''
    \newline
    European Network on High Performance and Embedded Architecture and Compilation.
\end{EntriesTable}

%%%%%%%%%%%%%%%%%%%%%%%%%%%%%%%%%%%%%%%%%%%%%%%%%%%%%%%%%%%%%%%%%%%%%%%%%%%%%%%

%%%%%%%%%%%%%%%%%%%%%%%%%%%%%%%%%%%%%%%%%%%%%%%%%%%%%%%%%%%%%%%%%%%%%%%%%%%%%%%
\section{Teaching}

\subsection{Undergraduate Modules}

\begin{EntriesTable}
   \Duration{2017}{\Ongoing}  &
   SCC365: Advanced Networking. \LUSCC. \newline
Research-informed Y3 optional module on computer network systems (20-60
students). \newline
Contribution: I deliver 5 lectures and lead all lab sessions. I was the module
convenor between 2017-2020. I Organize invited industry talks (BT, BBC, Google).\newline
Design \& Development: I redesigned the module structure in 2017 and aligned it
with the latest network research advancements. I developed a new 
lab and coursework framework in 2019, to improve repeatability and
user-friendliness, in response to student feedback.
    \\
    \Duration{2016}{\Ongoing} &
    SCC.150: Digital Systems. \LUSCC.
    \newline
Introductory Y1 module on Computer Architecture, Assembly programming, and
debugging (150-400 students).
    \newline
Contribution: I deliver all lectures between weeks 8 and 25, overlook lab organization and
manage 2 STAs. I am the
Module convenor since 2020.\newline
Design \& Development: I developed automated marking tools to
support coursework marking and cope with student number increases. 
I update annually the module structure to improve student engagement and improve
engagement. I developed a blended learning format for SCC.150 lectures during
the COVID pandemic, to improve student online engagement.   
\end{EntriesTable}


%%%%%%%%%%%%%%%%%%%%%%%%%%%%%%%%%%%%%%%%%%%%%%%%%%%%%%%%%%%%%%%%%%%%%%%%%%%%%%%
\subsection{P\lowercase{h}D supervision}

\begin{EntriesTable}
    \Duration{2019}{\Ongoing} &
    P. Alcock (Part-time) - \LUSCC
    \newline
    Multi-domain intent-based networking (Other Advisors: N. Race)
    \\
    \Duration{2019}{\Ongoing} &
    A. Magzoub (Part-time) - \LUSCC
    \newline
    Autonomic Networking: A knowledge-plane approach (Other Advisors: D. Hutchison)
    \\
    \Duration{2019}{\Ongoing} &
    A. Althobaiti (thesis submission by 1/2023) - \LUSCC
    \newline
    AI-based Energy theft detection (Other Advisors: A. Marnerides)
    \\
    \Duration{2019}{\Ongoing} &
    L. Hill - \LUSCC
    \newline
    Resilient networked systems using VNF (Other Advisors: D. Hutchison)
    \\
    \Duration {2018}{\Ongoing} &
    W. Fantom (Part-time) - \LUSCC
    \newline
    Network DevOps (Other Advisors: N. Race)
    \\
\Duration{2017}{\Ongoing}  &
  A. Jung (thesis submission by 1/2023) - \LUSCC
  \newline
  Cross-layer VM optimization (Other Advisors: D. Hutchison)
    \\
    \Duration{2017}{\Ongoing} &
    N. Hart (co-Advising) - \LUSCC
    \newline
    Programmable BGP (Other Advisors: D. Hutchison, N. Race)
\end{EntriesTable}

\subsection{Master Supervision}

\begin{EntriesTable}

\Duration{2021}{2022}  &
    R. Mathur, D. Pearce (MSci), A. Manzoor, W. Huang (MSc Sec)
\\
\Duration{2020}{2021}  &
E. Coterall, A. Piperides (MSci), D. Kantharow, R.
Ferguson  (MSc Sec)
\\
\Duration{2019}{2020}  &
    J. Hymes (MSci), R. Yates (MSc Data
    Science)
\\
\Duration{2018}{2019}  &
    N. Rutherford (MSci), A. Haseeb (MSc Data Science)
\\
\Duration{2017}{2018}  &
    A. Mofet (MSc Data Science)


\end{EntriesTable}

\subsection{PhD Examiner}
\begin{EntriesList}
    
Rafael Silva Guimarães -- ``Cross-layer programmability for expressive and agile
orchestration across heterogeneous resources'', External Examiner, Universidade Federal do Espírito Santo, 06/2021. \\
Jon Vidler -- ``Non-Linear Process Communication'', Internal Examiner, Lancaster
University, 2/2020. \\
Cornelius Toh Dong Tou -- ``Indoor positioning using semi-supervised fingerprint
building technique, based on crowd-sourced information'', External Examiner.
Sunway University, 3/2019.

\end{EntriesList}


\section{Engagement}

\subsection{Campus Administrative roles}

\begin{EntriesTable}
     \Duration{2022}{\Ongoing} & Part-I Tutor, \LUSCC. \newline
Part-I plagiarism officer. Engge with students Reps and interface them with
Part-I module teaching teams. \\
\Duration{2018}{2020} & Departmental committee on Wellbeing, \LUSCC.  \newline
Developed the post profile. Created online resources and organized workshops on
student wellbeing (exam support, coping with stress, signpost students to
university services.).\\
    \Duration{2018}{2021} & Part-I exam office, Part-II re-sit officer.\newline 
Developed an internal exam paper moderation mechanism to improve Part-I quality
assurance. Developed an SCC plan to ensure successful  exam
delivery, during COVID lockdowns. Developed online material and workshops on
digital marking.  \\
\Duration{2018}{\Ongoing} & Postgraduate Research committee member (Networking
group). \newline
I have organized annual meetings and offered research advice for more than 10
SCC PhD students.  \\
     \Duration{2017}{2018} & LU-Goenka SCC scheme director.  \newline
Represented SCC in progression boards. Reviewed annual ATP reports and provided input
to the SCC teaching committee.  
     \end{EntriesTable}

\subsection{Campus Workshops \& Short Courses Organization}

\begin{EntriesTable}
    \Year{2022}  &
    Welcome week lecture series: Welcome Week introduction to Part-I students. 
    \textit{\LUSCC, \LU}
    \\
    \Year{2019}  &
   Student wellbeing: Lecture series on mental wellbeing for UG students.
    \textit{\LUSCC, \LU}
    \\
    \Year{2018} &
    Networking group seminar series: Lecture series on netrwork and systems research
    methodologies.
    \textit{\LUSCC, \LU}

\end{EntriesTable}

\subsection{Business Collaborations}

\begin{EntriesTable}
    \Duration{2019}{\Ongoing} &
    Intent-based network management and automation, {\it with BT}. \newline
Collaborate with BT Digital Infrastructure and BT Global to develop several Proof-of-Concept
demonstrators for network automation, based on intent-based management
technologies. 
    \\
    \Duration{2018}{\Ongoing} &
    Wayfinder: cross-layer cloud optimization {\it with Unikraft}. \newline
Develop an open platform that use AI/ML techniques to discover
the optimal configuration (OS and application) for a cloud virtual machine, with
minimal user input. \\
    \Duration{2018}{\Ongoing} &
Ukmon: lightweight network monitoring infrastructure using unikernels {\it with
NEC Europe}. \newline
Develop an energy-efficient service-oriented network monitoring service using
unikernel-based network probes. \\
    \Duration{2017}{\Ongoing} &
    MvCDN: Shared edge-based CDN service architecture.
    Infrastructures {\it with BT and BBC}\newline
Design and deploy an edge-based video-delivery service for BBC live streams in
the BT production network. The service is currently operational in a local
telephony exchange in London and serves live BBC content to more than  100 BT customers.
    \\
\end{EntriesTable}


\subsection{Public \& Community Presentations}

\begin{EntriesTable}
    \Year{2022} & N. Race, \Me, S. Cassidy,  \emph{Spotlight on the future of networks II}.  Organize an NG-CDI expo event at BT Adastral Park.
    \\
    ~ & N. Race, \Me, S. Cassidy,  \emph{Spotlight on the future of networks}.  
    Organized an open online event to promote the finding of the NG-CDI project to BT stakeholders.
    \\
 ~ & 
    \Me, 
    Intent-driven network testing and monitoring.
    \emph{BT Thought Leadership series on Next Generation Converged Digital Infrastructure}
    \\
   ~ & 
    \Me.
    Assured Automation with ​ Network Testing \& Monitoring.
    \emph{Software-isation: The Challenges of Software Quality for ICT Intense Industries – Refining the Research Focus -- UK5G Showcase}
    \\
    \Year{2020} &
    N. Wang, \Me.
    Intent-Based Networking.
    \emph{BT Thought Leadership series on Next Generation Converged Digital Infrastructure}
    \\
    \Year{2019} &
    S. Simpson, P. Mccherry, A. Magzoub, A. Farshad, \Me \xspace
    A WIM Plugin for DataPlane Broker (DPB) (PoC Demo).
    \emph{7th OSM hackfest},
    Patras, Greece.
    \\
    \Year{2018} &
    \Me ,
    Resilient service orchestration.
    \emph{TOUCAN Industrial Schowcase}
    London, UK.
    \\
    \Year{2015} &
    N. Zilberman, G. Antichi, \Me ,
    Open Hardware Networking - Tutorial.
    \emph{SIGCOMM 2015}
    Imperial College London, UK.
    \\
    ~ &
    N. Zilberman, G. Antichi, \Me,
    Open Source Networking - Tutorial.
    \emph{Netsoft 2015}
    University College London, UK.
\end{EntriesTable}


\subsection{Reviewer}
IEEE Infocom (2020)

TPC for the European Workshop on SDN (EWSDN) (2012-­2015)

TPC for the International Teletrafic Conference (2015)

IEEE Conference on Computer Communications and Networks  (2015)

IEEE/ACM Transactions on Networking

Elsevier Computer Networks Journal

IEEE Communications Letters

IEEE Transactions on Network Service and Management


\subsection{Standardization}
Member of the Autonomous Network group, TeleManagement Forum (TMF). Responsibilities:
\indent {\it Network intent modeling, network automation standardization.}\\

Member of the European Telecommunications Standards Institute (ETSI) Network
Function Virtualization (NFV) Special Interest Group. Responsibilities:
\indent {\it PoC development, orchestration model design.} \\

\subsection{Open-Source Tools and Projects}

\begin{EntriesTable}

  \Duration{2020}{\Ongoing} &
  \textbf{Unikraft}
  --
  \href{https://github.com/unikraft}{github.com/unikraft}
  \newline
  A modular unikernel OS project, \textit{Responsibilities: Maintain monitoring appliances}.
  \\

  \Duration{2018}{\Ongoing} &
  \textbf{Data Plane Broker}
  --
  \href{https://github.com/DataPlaneBroker}{github.com/DataPlaneBroker}
  \newline
  A WAN management platform, \textit{Responsibilities: OSM Driver maintainer}.
  \\

   \Duration{2012}{\Ongoing} &
  \textbf{OFLOPS}
  --
  \href{https://github.com/oflops-nf/oflops-sume}{github.com/oflops-nf/oflops-sume}
  \newline
  A hardware OpenFlow switch benchmark platform.
  \textit{Responsibilities: Core developer}.
  \\
  \Duration{2014}{2018} &
  \textbf{Mirage Unikernel OS}
  --
  \href{https://www.mirage.org}{www.mirage.org}
  \newline
  An OCaml Unikernel OS project,
  \textit{Responsibilities: OpenFlow support developer}.
 \end{EntriesTable}

\section{Other Information - Feedback Evaluation}

\begin{table}[h]
\begin{tabular}{|c|c|c|c|}
\hline
Module & 2019-2020 & 2020-2021 & 2021-2022 \\
\hline
SCC.150 & 3.52 (response: 8\%) & 3.97 (response: 16\%) & 3.52 (response: 10\%) \\
\hline
SCC.365 & 4.30 (response: 22\%) & 3.86 (response: 12\%) & 4.00 (response: 18\%) \\
\hline
\end{tabular}
\caption{Overall student satisfaction and response rate for SCC.150 and SCC.365
modules for the last three academic years.}
\end{table}

\end{document}
